\documentclass{beamer}
%Information to be included in the title page:
\usetheme{Boadilla}
\definecolor{MITRed}{RGB}{163,31,52}
\definecolor{MITGray}{RGB}{138,139,140}
\usecolortheme[named=MITRed]{structure}

\setbeamertemplate{navigationsymbols}{}
\setbeamertemplate{headline}{\hfill\includegraphics[width=1.5cm]{./MIT-logo-red-gray.png}\hspace{0.4cm}\vspace{-1.2cm}}
\beamertemplatenavigationsymbolsempty
\setbeamertemplate{blocks}[default]
\setbeamercolor{item projected}{bg=MITGray}

\usepackage{bbm}
\usepackage{amsmath} % for the equation* environment

% \AtBeginSection[]
% {
% 	\begin{frame}
% 		\frametitle{Outline}
% 		\tableofcontents[currentsection]
% 	\end{frame}
% }
% \AtBeginSubsection[]
% {
% 	\begin{frame}
% 		\frametitle{Outline}
% 		\tableofcontents[currentsection,currentsubsection]
% 	\end{frame}
% }

\begin{document}
	\title[]{Finding the Sufficient Condition for the Formulation of Lobbying Industry}
	\subtitle{Simulative Experiment Using Multi-Agent Multi-Armed Bandit Model}
	\author[Suyeol Yun]{Suyeol Yun}
	\institute[MIT]{Massachusetts Institute of Technology}
	\date{Dec 8, 2022}
	\frame{\titlepage}
	\section{Background}

	\begin{frame}{Motivations}
		\begin{itemize}
			\item What is lobbying?
			\begin{itemize}
				\item What do lobbyists do?
				\item Buying vote? 
				\item Infleunce policy or legislators?
			\end{itemize}
		\end{itemize}
		\begin{itemize}
			\item Answer this question by
			\begin{itemize}
				\item Finding the sufficient condition that makes lobbying industry. 
				\item Search for the cause that makes the clients to hire lobbyists.
				\item By simulation experiment.				
			\end{itemize}
		\end{itemize}
	\end{frame}

	\begin{frame}{What is lobbying?}
		\begin{itemize}
			\item Yun and Preston (2022) define lobbying as \textbf{delegated information acquisition} process.
			\begin{itemize}
				\item Interest groups in lobbying industry hires lobbyists to acquire information about the policy and legislators.
				\item Lobbyists are the agents that acquire information about the policy and legislators on behalf of their clients.		
			\end{itemize}
			\item What kind of information are they acquiring?
			\begin{itemize}
				\item Information that can be used to maximize the political economic goal of the interest groups.
			\end{itemize}
		\end{itemize}
		% \centering	\includegraphics[scale=0.7]{./images/balancing.png}
	\end{frame}


	\begin{frame}{How to Acquire Information?}
		\begin{itemize}{
			\item By \textbf{interaction} with legislators.
				\begin{itemize}
					\item Campaign contribution
					\item Meeting with congressional staffers, etc.
				\end{itemize}
			\item How to model this interaction?
				 \begin{itemize}
					\item By using Multi-Armed Bandit (MAB) problem.
				\end{itemize}}
		\end{itemize}
	\end{frame}

	\begin{frame}{What is Multi-Armed Bandit\footnote{Bandit is pejorative name for slot machine because it empties players' pocket} (MAB) problem?}
		\begin{itemize}
			\item Formulate the exploration-exploitation dilemma problem.
			\begin{itemize}
				\item Formulate the \textbf{exploration-exploitation} dilemma problem.
				\item Assume that there are $K$ possible choices (called ``arms'') for the agent to make. 
				\item Each arm has a reward probability $P_k$ that is unknown to the agent.
				\item Whenever the agent chooses an arm, it receives a reward $r_k$ with probability $P_k$.
				\item The agent choose one of the arms for $T$ times and tries to maximize the total reward.
				\item As the agent sequentially choose the arms, the agent builds his own estimate of the reward probability $P_k$ of each arm.
				\item In this scenario, the agent keep facing the \textbf{exploration-exploitation} dilemma whether to choose the best rewarding arm based on his current estimates (so called ``exploitation'') or to try another arm to improve the current estimates because the current estimates can be biased (so called ``exploration'').
			\end{itemize}
		\end{itemize}
	\end{frame}

	\begin{frame}{MAB Problem of Interest Groups}
		\begin{itemize}
			\item Interest groups face the Multi-Armed Bandit problem when they participate information acquisition process in legislative space.
			\item There are $K$ number of legislators that interest groups can interact with. 
			\item Each legislator has a reward probability $P_k$ that is unknown to interest groups.
			\item Interest groups has a limited budget and time constraint to interact with legislators. (Model it as $T$ times of interaction)
			\item How to balance between exploration and exploitation to find the best fit legislator within this limited chances of interactions? (MAB problem)
		\end{itemize}
	\end{frame}

	\begin{frame}{Formulation of MAB Problem of Interest Groups} 
		\begin{itemize}
			\item There exists $C \in \mathbb{N}$ number of categories of interest.
			\item Assume there are $j \in J$ number of interest groups with $\phi(j): J \rightarrow C$ which represents unique category of interest $\phi(j)$ for each interest group.
			\item There are $K$ number of legislators with $P_k$ reward distribution.
			\item $P_k$ is modeled as a \textbf{Categorical distribution} with $C$ number of categories. $P_k$ is parameterized by $\mathbf{p_k} = [p_k^{(1)}, p_k^{(2)}, \hdots, p_k^{(C)}]$ where $p_k^{(i)} \in [0,1], \sum{p_k^{(i)}}=1$ with the support of $x_k \in \{1,2,\hdots,C\}$.  
			% (In other words, $X_k \sim \operatorname{Categorical}(\mathbf{p_k})$ where $X_k$ represents the category of authority of the $k$th legislator.)
			(In other words, )
			\item Whenever an interest group $j$ interact with a legislator $k$, they receive $c \in \{1, 2, \hdots, C\}$ sampled from $P_k$.
			\item  Each interst group $j$ gets reward of $r_j^{kn} = \mathbbm{1}(x_j^{kn} = \phi(j))$ when $j$ choose legislator $k$ at time $t$ and sampled $x_j^{kn}$ from $P_k$. 
			\begin{itemize}
				\item In other words, interest group $j$ with $\phi(j) = c$ gets reward of $1$ when they sample $c$ from interaction with legislator $k$.
			\end{itemize}
		\end{itemize}
	\end{frame}

	\begin{frame}{How to solve MAB problem?}
		\begin{itemize}
			\item Use \textbf{Thompson Sampling} algorithm.
			\begin{itemize}
				\item Each interest group $j$ has their own prior belief over $\mathbf{p_k}$.
				\item Keep updating this prior belief over $\mathbf{p_k}$ using the sampled observations from the interaction with legislators.
				\item The Dirichlet distribution is the conjugate prior of the categorical distribution.
				\begin{itemize}
					\item $f(\mathbf{p_k} | \mathcal{O}) \propto \mathcal{L(\mathcal{O | \mathbf{p_k}})} f(\mathbf{p_k})$ where $\mathbf{p_k} \sim \operatorname{Dirichlet(\mathbf{D})}$ with $\mathbf{D} = [d_1, d_2, \hdots, d_C] \in \mathbb{R}^C $ and $\mathcal{O}$ represents sampled observations from the interaction with legislators, i.e. $\mathcal{O} = [x_{j, t=1}^{k_1}, x_{j, t=2}^{k_2}, x_{j, t=3}^{k_3}, \hdots]$ where $x_{j, t}^{kt}$ is an observation sampled from the interaction with the legislator $k$ at time $t$. 
					\item This is a $C$ dimensional generalization of the Beta conjugate with Bernoulli likelihood.
					\item As we did in Beta conjugate, we can update the prior belief by simply adding $1$ to $d_c$ when observe $x_j = c \in C$ at each step $t \in T$.
					\item In this way of Bayesian update, interest groups can systematically update their prior belief over $\mathbf{p_k}$ based on obsevations from interactions with legislators.
				\end{itemize}
				\item After update, choose the best rewarding legislator $k$ based on the samples from posterior distributions $\{f(\mathbf{p_k} | \mathcal{O})| k \in \{1, 2, \hdots, C\}\}$ for the next legislator to interact with.
				%  Intuitively, the posterior distribution is the internal beleif of the interest group over the categorical distribution of legislator $k$
			\end{itemize}
		\end{itemize}
	\end{frame}


	\begin{frame}{Design of simulative experiment}
		\begin{itemize}
			\item Pattern of specialization
			\item In which condition does the client hire the lobbyist?
			\item 
		\end{itemize}
		% \centering	\includegraphics[scale=0.7]{./images/balancing.png}
	\end{frame}
	
	

	\begin{frame}{Limitation}
		\begin{itemize}
			\item Different from real world
			\item 
		\end{itemize}

		% \begin{block}{}
		% 	{\large ``In an election campaign with many competing candidates, a candidate can try to appeal broadly to all voters equally or concentrate more narrowly on winning the support of minorities or special interest groups...
		% 		\vskip5mm
		% 		\textit{Which strategy is better for winning an election?}'' (pg. 856)}
		% \end{block}
	\end{frame}

	\begin{frame}{Benifit of this theory}
		\begin{itemize}
			\item Encompasses influence theory as well because to influence, you first need to know influence whom.
		\end{itemize}
	\end{frame}


	\begin{frame}{Implications}
		\begin{itemize}
			\item provides criticism over Expertise vs. Connections (Is It Whom You Know or What You Know?
			An Empirical Assessment of the Lobbying Process)
		\end{itemize}
	\end{frame}



\end{document}


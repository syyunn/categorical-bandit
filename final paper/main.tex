\documentclass{article}
% if you need to pass options to natbib, use, e.g.:
%     \PassOptionsToPackage{numbers, compress}{natbib}
% before loading neurips_2020
% ready for submission
% \usepackage{neurips_2020}
% to compile a preprint version, e.g., for submission to arXiv, add add the
% [preprint] option:
%     \usepackage[preprint]{neurips_2020}
% to compile a camera-ready version, add the [final] option, e.g.:
%     \usepackage[final]{neurips_2020}
% to avoid loading the natbib package, add option nonatbib:
% \usepackage[nonatbib]{neurips_2020} % anonymous version
\usepackage[preprint]{neurips_2020}
\usepackage{graphicx}
\usepackage{subcaption}
\usepackage{amssymb}
\usepackage{float}
\usepackage{amsmath}
\usepackage{hyperref}
\usepackage{physics}
\usepackage{bbm}
\usepackage[utf8]{inputenc} % allow utf-8 input
\usepackage[T1]{fontenc}    % use 8-bit T1 fonts
\usepackage{hyperref}       % hyperlinks
\usepackage{url}            % simple URL typesetting
\usepackage{booktabs}       % professional-quality tables
\usepackage{amsfonts}       % blackboard math symbols
\usepackage{nicefrac}       % compact symbols for 1/2, etc.
\usepackage{microtype}      % microtypography
\title{Modeling 
Lobbying Industry with Multi-Agent Multi-Armed Bandit}
% The \author macro works with any number of authors. There are two commands
% used to separate the names and addresses of multiple authors: \And and \AND.
%
% Using \And between authors leaves it to LaTeX to determine where to break the
% lines. Using \AND forces a line break at that point. So, if LaTeX puts 3 of 4
% authors names on the first line, and the last on the second line, try using
% \AND instead of \And before the third author name.
\author{%
Suyeol Yun
% \thanks{
%   Use footnote for providing further information
%   about author (webpage, alternative address)---\emph{not} for acknowledging
%   funding agencies.
%   }
\\
Department of Political Science\\
Massachusetts Institute of Technology\\
Cambridge, MA 02139 \\
\texttt{syyun@mit.edu} \\
% examples of more authors
% \And
% Coauthor \\
% Affiliation \\
% Address \\
% \texttt{email} \\
% \AND
% Coauthor \\
% Affiliation \\
% Address \\
% \texttt{email} \\
% \And
% Coauthor \\
% Affiliation \\
% Address \\
% \texttt{email} \\
% \And
% Coauthor \\
% Affiliation \\
% Address \\
% \texttt{email} \\
}
\begin{document}
\maketitle
\begin{abstract}
What is lobbying?
To answer this question, 
I model the lobbying industry 
with a multi-agent multi-armed bandit problem.
Then I provide supportive evidence 
for the argument 
that explains lobbying 
as a delegated information acquisition process by simulation.
First, I show that 
interest groups can successfully explore 
a large legislative space
by delegating information acquisition to lobbyists.
Second, I show that 
lobbyists are hired by interest groups
when lobbyists have 
expert knowledge on a specific issue area 
across legislators.
These two resulsts  
provide supportive evidence
why interest groups 
have incentives to delegate information acquisition process to lobbyists
and why real-world lobbyists interact with both sides of the legislators in terms of interest of their clients.
\end{abstract}

\section{Introduction}
What is lobbying? 
This simple question has been answered in different literatures 
\citep{hall_deardorff_2006, 10.2307/43495360, 10.2307/3216842}
but there is no clear answer with agreement among scholars.
For example, \textit{vote-buying} theory argues 
that lobbying is to buy votes in legislation \citep{grossman}. 
However, this theory doesn't explain 
why the average price of vote is too cheap compared 
to the expected return of lobbying \citep{10.2307/3216842}.
As another example, \textit{persuasion} 
theory argue that lobbying is to persuade 
legislators to change their positions on legislation close to 
the position of the lobbyists
\citep{truman, Bauer2017, milbrath1984washington}.
However, this theory doesn't 
explain why lobbyists lobby 
already like-minded legislators 
those who don't need any persuasions \citep{10.2307/2586303}.

Apart from those orthodox theories,
this work 
provides a supportive evidence 
to the theory 
that explains lobbying as a 
delegated information acquisition process.
According to this theory, interest groups 
have incentives to delegate information acquisition process to lobbyists.
By modeling the lobbying industry 
using a multi-agent multi-armed bandit problem,
I simulatively show that
interest groups can successfully explore 
a large legislative space by delegating information acquisition process to lobbyists.

In addition, 
I also show that 
lobbyists are hired by interest groups 
when lobbyists have
expert knowledge on a specific issue area across 
legislators by simulation.
This result explains 
why lobbyists interact 
with both sides of the legislators
in terms of interest of their clients
\citep{10.2307/2586303}.
In other words, 
lobbyists interact with 
every relevant legislator 
in order to acquire information
on the issue area of their clients
regardless of the legislator's position on the issue area.

% In summary,
% this work models the lobbying industry 
% using a multi-agent multi-armed bandit (MMAB) problem.
% The motivation of using MMAB is 
% based on the fact   


% provides two different simulation results.
% First result shows why interest groups 
% have incentives to delegate information acquisition process to lobbyists.
% Second result shows why real-world lobbyists 
% interact with both sides of the legislators
% in terms of interest of their clients.



% In summary, 
% this work provides two different simulation results.
% First, I show that lobbying is a delegated information acquisition process.
% Second, I show that 





%  , aims to simulatively prove \textit{Reverse Information Theory on Lobbying (RIF)} \citep{rif} which explains lobbying as a delegated information acquisition process.
% According to RIF, every legislator has a different level of authority over different issue areas but clients lack of understanding which legislator has which authority over which issue areas.
% Therefore, they hire lobbyists or lobbying firms to acquire information related to their business.
% For example, Waymo LLC., an autonomous vehicle company reported they hired a lobbying firm, Holland \& Knight LLP., to acquire information on Build Back Better Act provisions on autonomous vehicles\footnote{\url{https://lda.senate.gov/filings/public/filing/0fe5096c-adc4-486a-93fe-7d8b99b823b5/print/}}. 
% To Waymo, keep tracking the development of Build Back Better Act will be costly and lack an expertise on analyzing complex legal articles and complicated politics behind it. Therefore, they hire highly specialized lobbying firm who has its own expertise on policy development on the issue area of autonomous vehicle. The lobbying firm Holland \& Knight also has NAVYA Inc., a self-driving solutions company as its client along with General Motors (GM) and Toyota. Seeing this, we can infer that Holland \& Knight has a specialized knowledge on legislative space on automobile industry and companies hire them as their lobbying firm to delegate their information acquisition process.

% % Acquisition of information on legislative space requires frequent interaction with legislators and this interaction is costly.
% RIF \citep{rif} argues that a group of clients who share the similar interests has an incentive to delegate their information acquisition process to a certain lobbying firm which has expertise on their issue areas of interest.
% To support this perspective, this paper models the lobbying industry as a multi-agent multi-armed (MAMA) bandit problem. 

% To model the U.S. lobbying industry as MAMA bandit problem, I model legislators as arms and companies as agents. 
% This is an abstraction of U.S. lobbying industry where the companies explore the best rewarding legislators related to their business. However, they don't know which legislators give the best reward to them \textit{a priori}.
% Rather, they need to deterimine how strategically explore and interact with which legislator to maximize their rewards.

% In addition, this paper models the rewards distribution of arms as categorical distribution to incorporate varying level of authorities of legislators over different issue areas.
% For example, Sen. Gary Peters may have a higher level of authority on issues related self-driving cars compared to the issues related to aviation because he's a chair of Senate sub-committee on surface transportation.
% Therefore, we can expect Waymo will acquire more reward from interacting with Gary Peters compared to interacting with Sen. Kyrsten Sinema who's a chair of Senate sub-committee on aviation.
% To model this situation, this paper assumes each agent has its own predefined categorical distribution and gets the highest reward when it matches with the arm's categorical distribution. This agent-specific categorical distribution represents the agent's unique preference over different issue areas.

% This paper conjectures incentive of hiring lobbying firm increases as the size of the exploration space increases.
% For example, if there exists a large number of legislators, companies will be more incentivized to hire lobbying firms to delegate their exploration due to the large space to explore. 
% Similarly, if there's a large number of categories for the reward distribution of legislators, companies will be more incentivized to hire lobbying firms to lower the cost of exploration. 
% To prove this conjecture simulatively, this paper incorporates a special multi-agents setting which allows agents to delegate their exploration to other agents. In detail, after every round of playing with arms, 
% agents can choose to delegate their exploration to other agents who have already explored the better reward to them. 
% To enable this, 
% % this paper adopts a contextual bandit where every agent gets the context of how much reward other agents get from interacting with which arms. Then by using this context, agents can decide whether to delegate their exploration to other agents for next round.
% % If the agent decides to delegate its exploration, it will get the reward following the action of the agent to whom it delegates its exploration.
% this paper provides a special arm to each agent on every round. Except the initial round, all agents get provided with the best rewarding arm to them from the pool of learned arms' distribution of all agents. 
% This arm represents "delegation" and agents who chose that delegating arm will get the reward following the action of the agent who is the master of that arm. However, since this delegation always give the benefit to agents,
% the reward from this special arm will be discounted with proper hyperparameter. Then total sum of discounted rewards will be conferred to the delegated agents. This resembles the situation where the lobbying firm is the master of the delegation arm and the client is the agent who delegates its exploration to the lobbying firm. Then the discount factor represents the cost of hiring the lobbying firm. 
% % and the lobbying firm collects its fee from all clients respectively.

% % This arm is a categorical distribution which is the same as the agent's own categorical distribution.


% In summary, this paper models the U.S. lobbying industry with categorized bandit problem where agents can delegate their exploration to other agent. By doing so, 
% this paper expects to simulatively prove that more number of agents tend to choose delegation as the number of legislators and the number of categories increase, which corresponds to the situation where agents have more large space to explore. 





% Since Multi-armed bandit describes an \textit{exploitation} and \textit{exploration} trade-off in sequential decision making. Multi-armed bandit is a metaphor of a reward-seeking agent
% who has multiple choices of actions (arms) and each action has a different payoff. This paper models the perspective in \cite{rif} as the multi-agents multi-armed (MAMA) bandit problem, where clients in the lobbying industry are agents who interact with legislators (arms) to acquire information about legislative space.
% By allowing those agents to delegate their future exploration to other agents, this paper expects to observe the emergence of lobbyists where the system converges to the stable existence of delegation.


\section{Information Acquisition: Multi-Agent Multi-Armed Bandit Problem}

In this section, I explain 
the relationship between 
the U.S. lobbying industry
and multi-agent multi-armed bandit problem.
% According to \cite{rif}, interest groups hire lobbyisyt to delegate their information acquisition process.
In U.S. lobbying industry, interest groups 
need to find which legislator
has the biggest influence over the issue area 
of their interest.
To do so,
they need to interact with different legislators 
to understand the expected rewards from each legislator.
However, total chances of interaction is limited 
because of the cost of interaction.
Therefore, interest groups 
need to design their 
exploration strategy
how to interact with different legislators
to maximize their expected rewards.
This situation can be modeled by multi-agent multi-armed bandit problem.
They decide 
whether to 
keep exploring or 
exploiting the best legislator according to their current knowledge
at each timestep.
% groups need to maximize their expected rewards from the exploration process.
Multi-armed bandit 
is a metaphor of a reward-seeking agent 
who has multiple choices of actions over multiple timesteps.
It formalizes the exploration–exploitation dilemma 
in sequential decision making\footnote{The problem is orginally 
proposed by \cite{Robbins1952SomeAO}
}.
Interest groups in the U.S. lobbying industry
need to maximize their expected rewards from the exploration process
by choosing the best strategy of exploration and exploitation of legislators at each timestep.
Therefore, 
I model the U.S. lobbying industry as multi-agent multi-armed bandit problem
and simulate to prove two conjectures. 
The First conjecture is  
the incentive of hiring lobbying firm increases
as the size of the exploration space increases.
The Second conjecture is
the incentive of hiring lobbying firm is higher in 
case that the lobbyist has expert knowledge on a 
particular issue area across different legislators
compared to the case that 
the lobbyist has expert knowledge 
on a particular legislator across different issue areas.


\subsection{\large{ 
Categorical Multi-Armed Bandit Problem}} \label{vanilla}

Multi-armed bandit problem 
assumes unknown reward distributions 
of each arm
which are unknown to agents.
In this section, I explain 
the \textit{categorical multi-armed bandit problem}\label{categorical}
which I used to model the 
environment of the U.S. lobbying industry.
Categorical multi-armed bandit problem
is a multi-armed bandit problem
that assumes the reward distribution of each arm to be categorical distribution.
Since legislators have varying level of authorities (or influences) 
over different issue areas,
I model this situation by categorical multi-armed bandit problem. There exists $|J|$ number of interest groups 
$\{1, 2, \hdots, J\}$, $|K|$ number of legislators $\{1, 2, \hdots, K\}$ 
, $|C|$ number of categories of interset $\{1,2, \hdots, C\}$ and
$|T|$ number of timesteps $\{0, 1, 2, \hdots, T\}$.  
Then each interst group $j \in J$ has its 
own category of interest $\phi(j) \in C$ and
each legislator $k \in K$ has a categorical distribution 
$\operatorname{Cat(C, \mathbf{p_k})}$ where
$C=\{1,2,\hdots, C\}$ and $\mathbf{p_k} = [p_k^{(1)}, p_k^{(2)}, \hdots, p_k^{(C)}]$
which represents the support of the distribution
and the vector of probabilities for each category of interest respectively.
Whenever an intesrest group $j \in J$ interact with the $k$-th legislator, they
sample $X_k \in C$ from $\operatorname{Cat(C, \mathbf{p_k})}$.
If the sampled category $X_k$ matches with the category of interest of the interest group, i.e. 
$X_k = \phi(j)$,
it gets the reward of $r_j^k = \mathbbm{1}(X_{k} = \phi(j))$. 
In other words, 
interest group $j \in J$
get the reward of $1$
when the sampled category $X_k$ matches with the category of interest of the interest group $\phi(j)$.
Otherwise, it gets the reward of $0$.
Therefore, interest groups 
need to find the best rewarding 
legislator in terms of their category of interest, i.e. 
$\operatorname{argmax}_{k \in K} p_k^{\phi(j)}$.


\subsection{\large{Thompson Sampling as a Bandit Strategy}}
There are many bandit strategies to solve the multi-armed bandit problem
such as $\epsilon$-greedy \citep{egrdy}, Upper Confidence Bound (UCB) \citep{ucb}, and Thompson Sampling \citep{tom}.
Those strategies are based on different exploration strategies.

Among those strategies,
Thompson Sampling is a Bayesian bandit strategy 
which models agent's prior and posterior belief over the reward distribution of each arm 
and systematically update it based on the observations.
I use Thompson Sampling as a bandit strategy in this work. 
This is because I can explicitly model the level of belief of each interest group over the reward distribution of each legislator.
This intuitively aligns with the real-world situation where 
interest groups have different level of understanding (which is belief) over the each legislator.



Let's assume that 
each interest group
hold a prior belief over the 
categorical distribution of each legislator.
Since Dirichlet distribution is conjugate prior of categorical distribution,
I model the prior belief of each interest group $j$ over the categorical distribution of the $k$-th legislator at timestep $t$, as
$\operatorname{Dir}(C, \mathbf{\alpha^t_{jk}})$ where 
$\mathbf{\alpha_{jk}} = [\alpha_{jk}^{t(1)}, \alpha_{jk}^{t(2)}, \hdots, \alpha_{jk}^{t(C)}]$.
At the beginning of the exploration process $t=0$, interest group $j$ 
holds the \textit{flat prior} belief over all legislators, i.e. $\mathbf{\alpha^{t=0}_{jk}} = [1, 1, \hdots, 1]\quad \forall k \in K$.
Then at each timestep $t\in T$, interest group $j$ choose the $k$-th legislator to 
interact with and sample $X_k^t \in C$ from $\operatorname{Cat(C, \mathbf{p_k})}$.
Then the interest group $j$ updates its prior belief by adding $1$ to $\alpha_{jk}^{t(X_k^t)}$. 
This yields the posterior belief at time $t+1$, $\operatorname{Dir}(C, \mathbf{\alpha}^{t+1}_{jk})$.
After update, the interest group $j$ chooses the next legislator $k_{t+1}$ 
by sampling $\mathbf{\hat{p}_{1j}}, \mathbf{\hat{p}_{2j}}, \hdots \mathbf{\hat{p}_{Kj}}$ from posterior beliefs 
from $\operatorname{Dir}(C, \mathbf{\alpha^{t+1}_{j1}})$,  $\operatorname{Dir}(C, \mathbf{\alpha^{t+1}_{j2}}), \hdots, \operatorname{Dir}(C, \mathbf{\alpha^{t+1}_{jK}})$ respectively.
$\mathbf{\hat{p}_{kj}}=[\hat{p}^{(1)}_{kj}, \hat{p}^{(2)}_{kj}, \hdots, \hat{p}^{(C)}_{kj}]$ is the vector of probabilities assigned for each category in case of legislator $k$.
Then $j$ choose $k_{t+1} = \underset{K}{\operatorname{argmax}}\{\hat{p}^{(\phi(j))}_{1j}, \hat{p}^{(\phi(j))}_{2j}, \hdots, \hat{p}^{(\phi(j))}_{Kj}\}$.
This means that the interest group $j$ chooses the next legislator $k_{t+1}$ 
based on samples from its own posterior belief.
Among the samples from posterior beliefs, $j$ only compares the 
dimension $\phi(j)$ of those samples,
probability of the category $\phi(j)$ for each legislator,
and chooses $k$ which has the highest probability of giving category $\phi(j)$ when sampled.
In this way, the interest group $j$ can systematically update
its prior belief and
balance the exploration and exploitation based on the 
randomness from the sampling. 
This strategy is called \textit{Thompson sampling} \citep{tom}.

\section{Large Search Space Problem in Lobbying Industry}\label{lss}

In this section, I provide a supportive evidence
that interest groups 
have incentives to delegate information acquisition process to lobbyists
to solve the \textit{large search space problem}. 
Starting from showing that 
interest groups fail to 
find the best rewarding legislator in
case of large search space,
I show that 
they can find the best rewarding legislator
by hiring lobbyists by simulation.

\subsection{\large{Simulation I: Small Search Space}}

First, I simulate the case of small search space.
In this simulation, I used $|K|=32, |C|=4, |J|=1\footnote{This is single agent setting}$ and $|T|=2000$.
Since I don't have 
have a good representation of
the categorical distributions of legislators,
I randomly generate a hypothetical categorical distribution of each legislator for simulation \footnote{By random sampling $\mathbf{p_k} = [p_k^{(1)}, p_k^{(2)}, \hdots, p_k^{(C)}]$ from $\operatorname{Dir}(C, \mathbf{\alpha})$ where $\alpha=[1,1, \hdots, 1]$,
I can generate hypothetical categorical distribution of each legislator $k$.}.
Besides, I use the normalized cumulative regret 
as the performance metric in the following simulations.
The cumulative normalized regret is 
defined as $\sum_t^T (p_{\phi(j)}^{\text{max}}-p_{a_t}^{\phi(j)}) / \sum_t^T(p_{\phi(j)}^{\text{max}}-p_{\phi(j)}^{\text{min}})$ 
where $p_{\phi(j)}^{\text{max}} = \operatorname{max}_{k \in K}\{p_1^{({\phi(j)})}, p_2^{({\phi(j)})}, \hdots, p_K^{({\phi(j)})}\} $,
$p_{\phi(j)}^{\text{min}} = \operatorname{min}_{k \in K}\{p_1^{({\phi(j)})}, p_2^{({\phi(j)})}, \hdots, p_K^{({\phi(j)})}\} $
and $a_t$ represent the choice of legislator taken by the interest group at time $t$.
Regret represents how much the agent could have done better in terms of the expected reward if it had chosen the best action.
Similarly, cumulative normalized regret 
is sum of regret over the entire time horizon $T$ normalized 
by the sum of upper bound of regret\footnote{I use this metric to compare the 
level of regret between different hyperparameter settings.
}. This metric is bounded in the range of $[0, 1]$ and 
the smaller the value is, the better the performance is.

Figure \ref{fig:rs_small} shows that 
normalized cumulative regrests of interest groups 
decrease as the number of timesteps increases.
The mean of cumulative normalized regrets at time $T$ from 
$10$ different random seeds\footnote{Each random seed  
corresponds to the different set of categorical distributions of legislators. 
} is $0.012$ with 
the variance of $5.98e-05$. 
This means that 
a interest group can successfully find the 
best rewarding legislator to interact in case of small search space.
In this scenario, search space consists of 
$|K| \times |C| = 128$ number of parameters 
which is relatively small compared to the case of 
actual lobbying industry.
% and interest groups can easily find the best legislator to interact with.
Figure \ref{fig:smallproba} shows that 
prior belief of the interest group well 
estimates the real probabilities of category of interest across legislators. 
It means that the mean of the posterior belief of the interest group corresponds to the real probability of category of interest across legislators.
In other words, the interest group successfully
reconstructed the categorical distributions of legislators within $2000$ times of interaction.
Therefore, figure \ref{fig:smallpropo} shows that the interest group 
successfully found the best legislator and keep interacting 
with the best one with the highest ratio.

\begin{figure}[h!]  
    \centering % Not {needed}
    \begin{subfigure}[b]{0.45\textwidth}
        % \centering
        \includegraphics[width=\linewidth]{images/rs_small_search_space.png}
        \caption{Cumulative normalized regret of interest groups in small search space with $10$ random seeds.}
        \label{fig:rs_small}
    \end{subfigure}
    % \hfill
    \begin{subfigure}[b]{0.5\textwidth}
        % \centering
        \includegraphics[width=\linewidth]{images/small_search_space/estimated_proba.png}
        \caption{$p_1^{(\phi(j))}, p_2^{(\phi(j))} \hdots p_K^{(\phi(j))} \text{ and } \hat{p}_1^{(\phi(j))}, \hat{p}_2^{(\phi(j))} \hdots \hat{p}_K^{(\phi(j))}$ for random seed $0$ at timestep $T$.}
        \label{fig:smallproba}
    \end{subfigure}
    %% leave a blank line to create a line break

    \begin{subfigure}[b]{1\columnwidth}
        \centering
        \includegraphics[width=0.5\textwidth]{images/small_search_space/propo_action_taken.png}
        \caption{The proportion of legislators being chosen by the interest group for random seed $0$.}
        \label{fig:smallpropo}
    \end{subfigure}
    \caption{Simulation in case of Small Search Space}
\end{figure}
    
\subsection{\large{Simulation II: Large Search Space}}\label{sim2}
In the previous simulaton, 
I used 
arbitrary small numbers for $|K|$ and $|C|$.
Now I use $|K|=112, |C|=26, |J|=1$ and $|T|=2000$.
$|K|=112$ is the average number of 
legislators to whom top 
$10$ lobbying firms campaign contribute in $2020$.
$|C|=26$ is the average number of
issue areas per clients in $2020$.
Both numbers are obtained from \textit{Lobbying Disclosure Act} data\footnote{Data is available through \url{https://lda.senate.gov/filings/public/filing/search/}}.
I expect the simulation with this configuration to be more realistic compared to the previous simulation.

Fig \ref{fig:rs_lg} shows that the mean of cumulative normalized regrets at time $T$ from 
$10$ different random seeds\footnote{Each random seed  
corresponds to the different set of categorical distributions of legislators. 
} is $0.48$ with the variance of $0.023$ which is much higher and larger compared to the previous simulation, which was $0.012$ with the variance of $5.98e-05$ respectively.
It implies that interset group fails to find the best rewarding legislator within $2000$ times of interaction.
This is because 
the search space is almost $23$ times larger than the previous simulation.
Due to the large search space, 
it's difficult for the interest group to 
explore enough and find the best legislator within the timesteps of $2000$.
Therefore, prior belief of the interest group fails to estimate 
the real world probabilities of legislators as shown in Fig \ref{fig:lgproba}.
This results in exploiting 
a worse legislator compared to the best one 
as shown in Figure $\ref{fig:lgpropo}$.
In this case, although the best legislator $k^*$ is the legislator $69$, 
the interest group keeps interacting with the legislator $90$ with the highest ratio. 

\begin{figure}[h!]
    \centering % Not {needed}
    \begin{subfigure}[b]{0.45\textwidth}
        % \centering
        \includegraphics[width=\linewidth]{images/large_search_space/rs.png}
        
        \caption{Cumulative normalized regret of interest groups in large search space}
        \label{fig:rs_lg}
    \end{subfigure}
    \hfill
    \begin{subfigure}[b]{0.45\textwidth}
        % \centering
        \includegraphics[width=\linewidth]{images/large_search_space/estimated_proba.png}
        \caption{$p_1^{(\phi(j))}, p_2^{(\phi(j))} \hdots p_K^{(\phi(j))} \text{ and } \hat{p}_1^{(\phi(j))}, \hat{p}_2^{(\phi(j))} \hdots \hat{p}_K^{(\phi(j))}$ for random seed $0$ at timestep $T$.}
        \label{fig:lgproba}
    \end{subfigure}
    %% leave a blank line to create a line break

    \begin{subfigure}[b]{1\columnwidth}
        \centering
        \includegraphics[width=0.5\textwidth]{images/large_search_space/propo_action_taken.png}
        \caption{The proportion of legislators being chosen by the interest group for random seed $0$.}
        \label{fig:lgpropo}
    \end{subfigure}
    \caption{Simulation in case of Large Search Space}
\end{figure}

\subsection{\large{Simulation III: Large Search Space with a Lobbyist}}\label{sim3}

In this simulation, I introduce a lobbyist into the multi-armed bandit setting explained in Section \ref{categorical}.
By showing that lobbyists can help interest groups to find the best legislators in the large search space, 
I provide a supportive evidence that lobbying can be explained as a delegated information acquisition process by simulation.


\subsubsection{\normalsize{Introducing Lobbyists into the Multi-Armed Bandit Setting}}\label{arml}
% Based on the  lobbying as a 
% delegated infomration acquisition process,
I conjecture 
that interest groups who share the same category of interest
can collaborate by sharing the same lobbyist. 
By doing so, they can concentrate their resources on finding the best legislators in the large search space.
In addition to the settings introduced in Section \ref{categorical},
I introduce a lobbyist $l \in L$ with its own set 
of prior beliefs $\operatorname{Dir}(C, \mathbf{\alpha_{lk}}) \text{ } \forall k \in K$ 
where $\mathbf{\alpha_{lk}} = [\alpha_{lk}^{(1)}, \alpha_{lk}^{(2)}, \hdots, \alpha_{lk}^{(C)}]$.
At each timestep $t$,
an interest group $j$ choose 
the next legislator $k_{t+1} = \underset{K}{\operatorname{argmax}}\left[\{\hat{p}^{(\phi(j))}_{1j}, \hat{p}^{(\phi(j))}_{2j}, \hdots, \hat{p}^{(\phi(j))}_{Kj}\}\cup\bigcup_{l\in L}\{\hat{p}^{(\phi(j))}_{1l}, \hat{p}^{(\phi(j))}_{2l}, \hdots, \hat{p}^{(\phi(j))}_{Kl}\}\right]$. 
This means that 
interest group $j$ chooses the best rewarding legislator based on the samples from the prior beliefs of all lobbyists including that of himself.
I assume complete information so that 
any interest groups can 
access to prior beliefs of all lobbyists at any timestep. 
In this setting, 
if the interest group $j$ 
chooses the next legislator $k_{t+1}$ 
based on 
the prior belief of lobbyist $l$, 
I update the prior belief of lobbyist 
$l$ with the sampled observation $X_{k}^{t+1} \sim \operatorname{Cat(C, \mathbf{p_k})}$
by adding $1$ to $\alpha_{lk}^{(X_k^{t+1})}$.
However, 
I don't update the 
prior belief of interest group $j$, $\alpha_{jk}^{(X_k^{t+1})}$.
This is because
if an interest group hires a lobbyist,
the lobbyist explore the legislative space on behalf of the interest group. 
This is because an interest group 
doesn't accumulate any experience when they hire a lobbyist.
Lobbyists interact with the legislators on behalf 
of the interest group and they accumulate the experience 
to themselves but not necessarily to the interest group.
In contrast, 
if an interest group choose the next legislator 
based on their own prior belief,
we update the prior belief of interest group $j$ as usual.
This modeling introduces the \textit{pros and cons} of hiring a lobbyist.
If an interest group hires a lobbyist, they 
may find the better legislator. However, 
they don't accumulate any experience to understand the reward distribution of the legislators
when they hire a lobbyist.


\subsubsection{Simulation : Large Search Space with a Lobbyist}

In this simulation, I maintain the size of the search space as same as the previous simulation 
by configuring $|K|=112, |C|=26, |T|=2000$.
However, I introduce a lobbyist $|L|=1$ and five interest groups with $|J|=5$. 
I assume that $5$ different interest groups
share the same category of interest. I choose $|J|=5$ based on the 
average number of clients per each issue code\footnote{
    \textit{Lobbying Disclosure Act} data provides $81$ number of issue codes such as 
    TRD (Trade) and TAX (Taxation) which are used to categorize intent of lobbying activities.
    } in \textit{Lobbying Disclosure Act} data in $2020$.
With this configuration, I expect 
the simulation to be more realistic.

In this simulation, the mean of cumulative normalized regrets at time $T$ from 
$10$ different random seeds\footnote{Each random seed  
corresponds to the different set of categorical distributions of legislators. 
} records $0.25$ with the variance of $0.007$ which are lower and smaller than those of the previous simulation, $0.48$, $0.023$ respectively. 
(See Fig \ref{fig:rs_lb}).
In addition, 
the interest groups found the best legislator as shown in Fig \ref{fig:lbpropo}.  
This is possible because multiple agents 
share the same lobbyist and collectively update 
the prior belief of the lobbyist. 
Fig \ref{fig:lbproba} 
shows that the interest group keep selecting into using 
the lobbyist's prior rather than its own prior. 
This is because
the lobbyist approximates the best rewarding legislator faster than interest groups.
Fig \ref{fig:esti_lb} shows that the lobbyist approximates 
the true reward distribution of the legislators closely after $2000$ timesteps. 
 
\subsection{\large{Lobbying as a Delegated Information Acquisition Process}}

Simulations in Section \ref{lss} 
provides a supportive evidence that 
lobbyists can be used to find the best legislators
when interest groups face large search space problem.
If interest groups share the 
same category of interest, they can form a collaborative
relationship via lobbyist 
to find the best legislator for their shared category of interest.
Regardless of whether interest groups intend to form a coalition or not, 
I argue that interest groups 
can form a coalition with other interst groups 
via lobbyists.
This finding is important 
because it provides a new perspective on 
lobbying not just as a delegated information acquisition process
but as a coalition formation process.
% \cite{hula} 
% argues that 
% formal or informal links across organizations reduce the costs of coalition formation in U.S. politics.
% Links formed by lobbyists can be one of the informal links that 

% Lobbyist 


% the growing tendency for groups to work within coalitions ex- pands the potential number of players exponentially. 



% For example, associations are one of the most 
% common type of registered 
% lobbyists in the United States. 
% With this modeling, we can understand that the 
% associations are coalition between members
% to successfully explore the large legislative space 
% by concentrating their resources to a single lobbyist, 
% which is association in this case.


\begin{figure}[h!]
    \centering % Not {needed}
    \begin{subfigure}[b]{0.45\textwidth}
        % \centering
        
        \includegraphics[width=\linewidth]{images/large_search_lobbyist/rs.png}
        
        \caption{Cumulative normalized regret of interest groups in large search space with lobbyist}
        \label{fig:rs_lb}

    \end{subfigure}
    \hfill
    \begin{subfigure}[b]{0.45\textwidth}
        % \centering
        \includegraphics[width=\linewidth]{images/large_search_lobbyist/freq.png}
        \caption{Frquency of using lobbyist. $0$ represents using lobbyist and $-1$ represents 
        \textit{not} using lobbyist, relying on interst groups own prior.
        }
        \label{fig:lbproba}
    \end{subfigure}
    %% leave a blank line to create a line break

    \begin{subfigure}[b]{0.45\textwidth}
        % \centering
        
        \includegraphics[width=\linewidth]{images/large_search_lobbyist/lb_esti.png}
        
        \caption{$p_1^{(\phi(j))}, p_2^{(\phi(j))} \hdots p_K^{(\phi(j))} \text{ and } \hat{p}_1^{(\phi(j))}, \hat{p}_2^{(\phi(j))} \hdots \hat{p}_K^{(\phi(j))}$ of \textit{lobbyist} for random seed $0$ at timestep $T$.}
        \label{fig:esti_lb}

    \end{subfigure}
    \hfill
    \begin{subfigure}[b]{0.45\textwidth}
        % \centering
        \includegraphics[width=\linewidth]{images/large_search_lobbyist/best_arm.png}
        \caption{The proportion of legislators being chosen by the interest group for random seed $0$.}
        \label{fig:lbpropo}
    \end{subfigure}
    %% leave a blank line to create a line break

    \caption{Simulation in case of Large Search Space with Lobbyist}
\end{figure}

\section{Specialization of Lobbyists and Expert knowledge}

In this section, I find the reason
why lobbyists lobby both sides of legislators in reality.
This phenomenon that lobbyists lobbying both sides of legislators 
is regarded as a puzzle in the literatures because lobbyists don't need to 
lobby already supportive legislators according to the persuasion (informational) theories on lobbying \citep{10.2307/2586303}.


To do so, I compare the simulation results from the two differet types of expert knowledge of lobbyists.
In the section \ref{arml}, I assumed 
that each lobbyist $l$ has its own prior belief of legislators $\mathbf{\alpha_{l1}}, \mathbf{\alpha_{l2}}, \hdots, \mathbf{\alpha_{lK}} \in \mathbb{N}^C$ which are $K$ number of concentration parameter of dirichlet distributions.
It can be visualized as shown in Fig \ref{fig:mesh1}. 
A lobbyist can either have expert knowledge 
on a specific issue area across all legislators (blue in Fig \ref{fig:mesh1}) or
have expert knowledge on a specific legislator (red in Fig \ref{fig:mesh1})
across all issue areas. 
Here, 

I conjecture that lobbyists can be specialized 
in case lobbyists having 
expert knowledge on a specific issue area across all legislators
rather than having expert knowledge 
on a specific legislator across all issue areas.
Here, I use the word \textit{specialized} to mean that
lobbyists get hired by interest groups 
that matches the expert knowledge of lobbyists.
If the conjecture is true, 
lobbyists will lobby both sides of legislators
to acquire 
the expert knowledge on a specific issue area
across all legislators.

\begin{figure}[h!]
    \centering
    \includegraphics[width=0.6\textwidth]{images/pos.png}
    \caption{Prior belief of lobbyist is paramaterized by $K \times C$ matrix.}
    \label{fig:mesh1}
\end{figure}


% I provide a supportive evidence to this conjecture by comparing the simulation results 
% from two different types of expert knowledge of lobbyists.

\subsection{\large{Simulation Setting}}

In this simulation, 
I use the configuration of $|K|=112, |C|=26, |T|=2000, |J|=10$ and $|L|=2$. 
I maintain the size of search space as used in the section \ref{sim2} and \ref{sim3} by setting $|K|=112, |C|=26$.
Then partition $10$ interest groups 
into two groups, $5$ for each group and 
assume that the interest groups in the same group share the same category of interest 
while interest groups in different groups have different categories of interest $c_1$ and $c_2$.
Therefore, I add $|L|=2$ lobbyists to represent the 
category of interest of each group. I define a category of expert knowledge of a lobbyist $l$ as 
$\upsilon(l)$. For the two lobbyist $l_1$ and $l_2$, I assign $\upsilon(l_1) = c_1$ and $\upsilon(l_2) = c_2$ respectively.

\subsubsection{Simulation Setting For Expert Knowledge}
In this subsection, I describe the simulation setting for the two different types of expert knowledge of lobbyists.

First, to model the expert knowledge of lobbyists 
on a specific issue area across all legislators (blue in Fig \ref{fig:mesh1}),
I preset the prior belief of lobbyists to be close to the categorical distribution of legislators.
By setting $\alpha_{lk}^{(c)} = p_k^{(c)} \cdot |C| \cdot \delta $ and $\alpha_{lk}^{(-c)} = (|C| - \alpha_{lk}^{(c)}/(|C|-1) \text{ } \forall k$, 
I introduced expert knowledge on a specific issue area $c$ across all legislators.
This setting is a redistribution of the sum of concentration parameter of dirichlet distribution which is equal to $|C|$.
The distribution of the quantity $|C|$ across all categories determines the expectation over 
the parameter of the categorical distribution of legislator $k$. 
Therefore, if I set $\alpha_{lk}^{(c)} = p_k^{(c)} \cdot |C| \cdot \delta $ 
and distribute remaining quantity $|C| - \alpha_{lk}^{(c)}$ across all other $|C|-1$ number of categories ,
it makes the expectation of such dirichlet distribution for category $c$ to be eqaul to $p_k^{(c)}$.
In addition, by introducing $\delta \in [0,1]$ which is a temperature parameter to control the degree of expert knowledge,
I can gradually introduce expert knowledge on a specific issue area $c$.

Second, to model the expert knowledge of lobbyists 
on a specific legislator $k^*$ across all issue areas (red in Fig \ref{fig:mesh1}),
I set $\alpha_{lk^*}^{(c)} = p_{k^*}^{(c)}\cdot |C| \cdot \delta + \epsilon \text{  } \forall c$ where $k^* = \operatorname{argmax_K}\{p_1^{(c)}, p_2^{(c)}, \hdots, p_K^{(c)}\}$ .
$\epsilon$ is a small number to prevent concentration parameter from being zero when $\delta = 0$.
This means that first I find the $k^*$ which represent the best rewarding arm in terms of category $c$ and 
copy his/her parameter of categorical distribution to dirichlet parameter of lobbyist.
As same as the first setting, by introducing $\delta \in [0,1]$
I can gradually introduce expert knowledge on a specific legislator $k^*$ across all issue areas.

With this setting, I ran the simulation and measure 
the ratio of hiring matching lobbyist. 
Matching lobbyist is defined as the lobbyist who has the expert knowledge on the category of interest of interest groups.
For example, if the category of interest of interest group $j$ is $c_1$
and the expert knowledge of lobbyist $l_1$ is $c_1$, 
then $l_1$ is a ``matching'' lobbyist for interest group $j$.
Then the ratio of hiring matching lobbyist is defined as 
the ratio of hiring ``matching'' lobbyist over the total number of timesteps $|T|=2000$.

Since each group of interest groups has the shared category of interest $c_1, c_2$
and two lobbyist $l_1, l_2$ have the expert knowledge on $c_1, c_2$ respectively, 
I expect that the ratio of hiring ``matching'' lobbyist increases 
as the temperature parameter $\delta$ increases. 

\subsection{\large{Simulation Result}}

Figure \ref{fig:specialization} shows the simulation results.
The simulation is ran twice for two different types of expert knowledge of lobbyists.
The figure shows that only in case of expert knowledge on a specific issue area across all legislators (blue in Fig \ref{fig:mesh1}),
the ratio of hiring ``matching'' lobbyist is higher than $0.5$, which is a baseline of randomly selecting a lobbyist between two available lobbyists.
In case of lobbyist having expert knowledge on a specific issue area across all legislators,
ratio of hiring "matching" lobbyist converges to $1$ as the temperature parameter $\delta$ increases. 
However, in case of lobbyist having expert knowledge on a specific legislator across all issue areas (red in Fig \ref{fig:mesh1}),
ratio of hiring "matching" lobbyist never become explicitly higher than $0.5$.
This result indicates that lobbyists are specialized in lobbying industry
toward the direction of acquiring expert knowledge on a specific issue area across all legislators.
In other words, lobbyists are selected by 
interest groups only when they have the expert knowledge on a specific issue area across all legislators.
This is intuitively reasonable because lobbyists are 
a ``middleman'' between interest groups and legislators
who explore the legislative space on behalf of interest groups.
Since the meaning of exploration is to find the legislator 
who has the best expected retrun on a specific issue area,
lobbyists need to equip themselves with expert knowledge on a specific issue area across all legislators
which can provide ``comparison'' between legislators.
Otherwise, simply knowing one best legislator without knowing other legislators' expected return
is not enough to make a decision for interest groups to particularly hire such lobbyist.
Therefore, for lobbyists to survive in the market,
they need to accumulate expert knowledge on a specific issue area across all legislators.
This result explains a puzzle in the literatures why lobbyists lobby to both sides of legislators.

\begin{figure}[h!]
    \centering
    \includegraphics[width=1\textwidth]{images/Figure_1_scatter.png}
    \caption{Mean ratio of hiring ``matching'' lobbyist for two different types of expert knowledge of lobbyist.}
    \label{fig:specialization}
\end{figure}

\section{Conclusion: Limitations and Future Directions}

In this work, 
I modeled the lobbying industry as a multi-armed bandit problem.
By modeling lobbying industry as a simulative environment based on 
multi-armed bandit problem, I found that 
interest groups are facing large search space problem 
and they can overcome this by using lobbyists.
In addition, I found that lobbyists are hired 
by interest groups 
only when they have the expert knowledge on a specific issue area across all legislators.
This finding explains a puzzle in the literatures why lobbyists lobby to both sides of legislators.

Although this work shed a light on understanding lobbying industry, 
there are still many limitations in this work. First, this work 
uses a \textit{hypothetical} categorical distribution of 
legislators' influence over different issue areas.
Since this distribution is hypothetical,
finidngs from a few random seeds might be limited to 
extend to the real world.
Second, this work assumes each interest group has a unique 
category of interest and all categories are orthogonal to each other.
However, each 
interest group has multiple categories of interest and 
those categories are correlated to each other. 
Third, this work provides simulation results with 
very small number of interest groups and lobbyists. In reality,
there are more than 70,000 interest groups and 20,000 lobbyists interacting with each other.
Therefore, this work needs to be extended to a larger scale simulation 
to approximate the dynamics of lobbying industry more realistically.

Based on those limitations,
I plan to extend this work in the future. 
First, I plan to find the good representation of 
legislators' influence over different issue areas.
Recent development of 
\textit{reinforcement learning} 
provides multiple methodologies to estimate the value of each action (or policy\footnote{Policy is defined as a function that maps state to optimal actions. In other words, policy represents a planning about how to act under varying situations. In \textit{reifn\
 learning}, we keep evaluating the value of each policy and find the optimal policy.
 }) for different sitautions \citep{Sutton1998}.
By using those methodologies, 
I can estimate the value of different actions which is interaction with different legislators in this case.
If I model this value function 
as a function of interest group and legislator specific characteristics,
I can approximate more realistic reward function which accomodates the real world characteristics of interest groups and legislators.
Second, I plan to extend this work to a larger scale simulation. 
Since this simulation is based on ``shared enviornment'' between multiple agents, 
simple parallelization based on multi-processing doesn't work because it requires a \textit{shared} memory between
multiple agents. To overcome this, \textit{decentralized} multi-agent simulation method is needed.
In this method, each agent uses its own memory and relies on its own estimation of the behavior of other agents and the environment \citep{decen}.
In this way, agents can be simulated in parallel and the entire simulation can be extended to a larger scale.

In summary, this work provides 
a supportive evidence for a new perspective on understanding lobbying industry. 
In addition,
this work also provides a formal way of 
modeling lobbying industry which can be simulatively 
tested with different assumptions and parameters.
Since the components introduced in this modeling 
are replaceable with more realistic components,
this work is modular so that it can be extended based on 
future researches that provide more realistic components.
Although it's still controversial whether this kind of simulative research can 
be accepted as a valid knowledge about the real world,
more empirical studies based on findings from this kind of 
simulations may indirectly prove the 
usefulness and validity of this simulative research in the future.


% \ref{vanilla} models the payoff of each arm as Bernoulli distribution. However, this paper models the payoff of each arm as categorical distribution.
% This is to represent each legislator's varying level of authority over different issue areas. 
% % For example, senator Joe Manchin has a high authority
% % over the issue area of energy and natural re
% sources because he is the chair of the Senate Energy and Natural Resources Committee.
% % In comparison, senator Ron Wyden has a higher authority over the issue area of tax policy because he is the chair of the Senate Finance Committee.
% To model this varying level of authority over different issue areas, this paper models the payoff of each arm as categorical distribution. This scenario is recognized under the name of \textit{categorized bandit} and 
% \cite{catego} generalizes \textit{Thomposon sampling} \citep{tom} and introduces \textit{Murphy Sampling} to solve the categorical bandit problem. However, this solution doesn't consider the case of each agent having ordered preference over different categories.
% In the lobbying industry, it's common that clients have ordered preference over different issue areas. Therefore, the reward should be maximized when the ordered preference of each agent aligns with the categorical distribution of the legislator.
% % For example, \textit{Hyundai Motors, Co.}, a South Korea automobile manufacturer, was recently excluded from the U.S. government's environmental subsidy program for electric cars.
% % Since this exclusion is included in \textit{Inflation Reduction Act of 2022} and the bill was introduced by Rep. John Yarmuth, \textit{Hyundai Motors, Co.} has a huge incentive to interact with Rep. John Yarmuth to acquire information about the bill.
% In this context, \cite{NEURIPS2019_83462e22} introduces a new algorithm called \textit{CatSE}\label{cat} to solve the categorized bandit problem under the assumption of ordered preference of each agent. Therefore, this paper will use this algorithm to solve the categorized bandit problem.

% \subsection{Multi-Agent Multi-Armed (MAMA) Bandit Problem}
% % Since this paper aims to plausibly abstractize U.S. lobbying industry and observe the emergence of lobbyists by simulation, this paper allows the delegation of exploration to other agents. 
% % Now let's think about multiple bandit agents that can successfully update their posterior distribution by interacting with arms.

% Simply creating multiple agents doesn't make it a meaningful multi-agent problem because the agents are not interacting with each other.
% To plausibly abstractize the U.S. lobbying industry, this paper assumes a special institutional structure where agents can delegate their exploration to other agents.
% To implment this, this paper provides a special arm to each agent on every round. Except the initial round, all agents get provided with the best rewarding arm to them from the pool of learned arms' distribution of all agents. 
% This arm represents "delegation" and agents who chose that delegating arm will get the reward following the action of the agent who is the master of that arm. However, since this delegation always give the benefit to agents,

% the reward from this special arm will be discounted with proper hyperparameter. Then total sum of discounted rewards will be conferred to the delegated agents. This resembles the situation where the lobbying firm is the master of the delegation arm and the client is the agent who delegates its exploration to the lobbying firm. Then the discount factor represents the cost of hiring the lobbying firm.

% This paper expects that agents who have more biased preference over specific category will be eventually be the master of the delegation arm. This is because the agents who have more biased preference over specific category will be more likely to choose the arm that has the highest reward for them. 
% This represents the U.S. lobbying industry where the lobbyists are specialized in specific issue areas and the clients delegate their exploration to those specialized lobbyists.


% Therefore, this paper expects that the agents who have more biased preference over specific category will be eventually be the master of the delegation arm. This is because the agents who have more biased preference over specific category will be more likely to choose the arm that has the highest reward for them. Therefore, this paper expects that the agents who have more biased preference over specific category will be eventually be the master of the delegation arm. This is because the agents who have more biased preference over specific category will be more likely to choose the arm that has the highest reward for them. Therefore, this paper expects that the agents who have more biased preference over specific category will be eventually be the master of the delegation arm. This is because the agents who have more biased preference over specific category will be more likely to choose the arm that has the highest reward for them. Therefore, this paper expects that the agents who have more biased preference over specific category will be eventually be the master of the delegation arm. This is because the agents who have more biased preference over specific category will be more likely to choose the arm that has the highest reward for them. Therefore, this paper expects that the agents who have more biased preference over specific category will be eventually be the master of the delegation arm. This is because the agents who have more biased preference over specific category will be more likely to choose the arm that has the highest reward for them. Therefore, this paper expects that the agents who have more biased preference over specific category will be eventually be the master of the delegation arm. This is because the agents who have more biased preference over specific category will be more likely to choose the arm that has the highest reward for them. Therefore, this paper expects that the agents who have more biased preference over specific category will be eventually be the master of the delegation arm. This is because the agents who have more biased preference over specific category will be more likely to choose the arm that has the highest reward for them. Therefore, this paper expects that the agents who have more biased preference over specific category will be eventually be the master of the delegation arm. This is because the agents who have more biased preference over specific category will be more likely to choose the arm that has the highest reward for them. 




%and the lobbying firm.  collects its fee from all clients respectively.


% The rationale behind the multi-agent scenario is to observe how the agents interact with each other and this results in changes in the system.
% Since this paper aims to plausibly abstractize U.S. lobbying industry and observe the emergence of lobbyists by simulation, this paper allows the delegation of exploration to other agents. 
% For example, every round we add a special arm that represents "delegation" where it is the other agent's estimates over the payoff of an arm that provided best reward 
% To plausibly abstractize the U.S. lobbying industry and observe the emergence of lobbyists by simulation, this paper allows the delegation of exploration to other agents.


% show the emergence of lobbyists in a purely theoretical and simulated environment where no empirical data is involved. In other words,
% This paper designs the MAMA bandit scenario that is plausibly similar to the real world lobbying industry. In addition, this paper is motivated by \cite{rif} which argues that lobbying is
% to delegate information acquisition processes to lobbyists.
% I suspect the incentive of delegation emerges from the initial asymmetry of information between agents.
% In detail, at the initial stage, let's suppose an agent $A$ who interacted with a legislator who doesn't have a proper authority over the issue area where the agent $A$ is interested.
% However, at the same time, there could be another agent $B$ who interacts with a legislator who has a proper authority over the issue area where the agent $A$ is interested.
% Therefore, if we allow the agents to \textit{trade} their information, the agent $A$ can acquire the information about the issue area that he's interested in.
% For example, \cite{mama} simulatively shows that the agents acquire higher total rewards if they're allowed to trade their information in MAMA situations.
% In addition, if we allow agents to \textit{delegate} exploration, all agents are expected to score higher total rewards (pareto improvement). This is because a group of agents who share the same issue area of interest can explore more efficiently without duplicated interactions with poorly rewarding legislators when exploration is delegated to specific agents.
% Therefore, this paper aims to find the set of institutional features that can lead to the emergence of lobbyists in a simulative environment.
% Initial hunch is if we allow agents to delegate exploration, the system will converge to the equilibriums where the agents are grouped
% by their issue area of interest and the agents who are responsible for exploration will arise. However, it's important to incorporate some constraints for delegation of exploration between agents to make the system more realistic.
% For example, in the real world, clients hire lobbyists because it's cheap and efficient to delegate information acquisition processes to lobbyists.
% Therefore, this paper should include the set of parameters that controls the \textit{discount factor} for the cost of exploration when the agent delegates its exploration to other agents. In addition, this paper also should include
% some rules for the agents to determine whether to keep delegating exploration to other agents or not. In the real world, it's quite common that clients terminate the contract with lobbyists when they think it's not worth it to keep hiring them.
% This kind of decision will be possible to incorporate by including another $\mathrm{Bernoulli}$ arm that represents the decision of the agent to keep delegating exploration to other agents or not.
% By doing so, this paper will be able to interpret the posterior distribution of those $\mathrm{Bernoulli}$ arms as the probability of the agent to keep delegating exploration to other agents. Therefore, this will prove or disprove the stable existence of lobbyists in the system.

% \subsection{Purely Simulative Environment and Claim of the Paper}
% In this paper, I focus on the varying equalibrium across the different hyperparmeters that deterimines the number of legislators, number of categories and discount factors for delegation.
% % ratios of the number of legilsators to the number of clients.
% If the number of legislstors are very small, the incentive of delegation is weak because the exchaustive exploration is possible.
% However, if the number of legislators are sufficiently large, the incentive of delegation is strong because the exchaustive exploration is impossible
% and benefits from the delegation become more significant. To explore the emergence of lobbyists within the framework of exploitation and exploration dillemma with regard to the size of the exploration space, this paper purely relies on the simulation without involving any empirical data. 
% However, by observing the change of the equalibrium of the system with varying number of legilsators and issue areas, this paper aims to prove that lobbyists emerge when the number of legislators and number of issue categories are sufficiently large.



% \section{Summary of the Proposal and Future Directions}
 
% This paper aims to reproduce the emergence of lobbyists in a simulated environment.
% To do so, it's required to implement $(1)$ categorized bandit \citep{NEURIPS2019_83462e22} and $(2)$ institutional feature that enables delegation.
% After implementation, this paper will simulate the system and observe $(1)$ stability of the practice of delegation among agents and $(2)$ distributional characteristics of agents who keep delegated by other agents.
 
% The biggest motivation to adopt this simulative approach is to prepare a computational environment that can model the different behavioral and institutional features that different theories of lobbying highlight.
% % For example, \textit{persuasion} theories of lobbying are attacked under the empirical findings that people also lobby already like-minded legislators who don't need to be further persuaded.
% % However, I suspect this is because of the dynamically changing authority of legislators over time.
% % For the simplicity, this paper doesn't plan to model this dynamically changing authority of legislators over time.
% Once this simulative environment is prepared, we can gradually build up and test the different theories of lobbying in this simulation environment.
% Moreover, although this paper doesn't plan to involve any empirical data, this simulative environment can be equipped with the empirical data to more closely approximate the U.S. lobbying industry.
% For example, we can use the actual bill sponsor information to model the varying authorities of legislators over different issue areas. Also, we can use the actual lobbying data to model the varying preferences of clients over different issue areas.
% In conclusion, by preparing this computational environment, researchers will be able to test different theories on lobbying in a simulative environment. By doing so, this paper expects to facilitates agreement among scholars on answer to the question - "What is lobbying?".
% % The style files for NeurIPS and other conference information are available on
% % the World Wide Web at
% % \begin{center}
% %   \url{http://www.neurips.cc/}
% % \end{center}
% % The file \verb+neurips_2020.pdf+ contains these instructions and illustrates the
% % various formatting requirements your NeurIPS paper must satisfy.
% % The only supported style file for NeurIPS 2020 is \verb+neurips_2020.sty+,
% % rewritten for \LaTeXe{}.  \textbf{Previous style files for \LaTeX{} 2.09,
% %   Microsoft Word, and RTF are no longer supported!}
% % The \LaTeX{} style file contains three optional arguments: \verb+final+, which
% % creates a camera-ready copy, \verb+preprint+, which creates a preprint for
% % submission to, e.g., arXiv, and \verb+nonatbib+, which will not load the
% % \verb+natbib+ package for you in case of package clash.
% % \paragraph{Preprint option}
% % If you wish to post a preprint of your work online, e.g., on arXiv, using the
% % NeurIPS style, please use the \verb+preprint+ option. This will create a
% % nonanonymized version of your work with the text ``Preprint. Work in progress.''
% % in the footer. This version may be distributed as you see fit. Please \textbf{do
% %   not} use the \verb+final+ option, which should \textbf{only} be used for
% % papers accepted to NeurIPS.
% % At submission time, please omit the \verb+final+ and \verb+preprint+
% % options. This will anonymize your submission and add line numbers to aid
% % review. Please do \emph{not} refer to these line numbers in your paper as they
% % will be removed during generation of camera-ready copies.
% % The file \verb+neurips_2020.tex+ may be used as a ``shell'' for writing your
% % paper. All you have to do is replace the author, title, abstract, and text of
% % the paper with your own.
% % The formatting instructions contained in these style files are summarized in
% % Sections \ref{gen_inst}, \ref{headings}, and \ref{others} below.
% % \section{General formatting instructions}
% % \label{gen_inst}
% % The text must be confined within a rectangle 5.5~inches (33~picas) wide and
% % 9~inches (54~picas) long. The left margin is 1.5~inch (9~picas).  Use 10~point
% % type with a vertical spacing (leading) of 11~points.  Times New Roman is the
% % preferred typeface throughout, and will be selected for you by default.
% % Paragraphs are separated by \nicefrac{1}{2}~line space (5.5 points), with no
% % indentation.
% % The paper title should be 17~point, initial caps/lower case, bold, centered
% % between two horizontal rules. The top rule should be 4~points thick and the
% % bottom rule should be 1~point thick. Allow \nicefrac{1}{4}~inch space above and
% % below the title to rules. All pages should start at 1~inch (6~picas) from the
% % top of the page.
% % For the final version, authors' names are set in boldface, and each name is
% % centered above the corresponding address. The lead author's name is to be listed
% % first (left-most), and the co-authors' names (if different address) are set to
% % follow. If there is only one co-author, list both author and co-author side by
% % side.
% % Please pay special attention to the instructions in Section \ref{others}
% % regarding figures, tables, acknowledgments, and references.
% % \section{Headings: first level}
% % \label{headings}
% % All headings should be lower case (except for first word and proper nouns),
% % flush left, and bold.
% % First-level headings should be in 12-point type.
% % \subsection{Headings: second level}
% % Second-level headings should be in 10-point type.
% % \subsubsection{Headings: third level}
% % Third-level headings should be in 10-point type.
% % \paragraph{Paragraphs}
% % There is also a \verb+\paragraph+ command available, which sets the heading in
% % bold, flush left, and inline with the text, with the heading followed by 1\,em
% % of space.
% % \section{Citations, figures, tables, references}
% % \label{others}
% % These instructions apply to everyone.
% % \subsection{Citations within the text}
% % The \verb+natbib+ package will be loaded for you by default.  Citations may be
% % author/year or numeric, as long as you maintain internal consistency.  As to the
% % format of the references themselves, any style is acceptable as long as it is
% % used consistently.
% % The documentation for \verb+natbib+ may be found at
% % \begin{center}
% %   \url{http://mirrors.ctan.org/macros/latex/contrib/natbib/natnotes.pdf}
% % \end{center}
% % Of note is the command \verb+\citet+, which produces citations appropriate for
% % use in inline text.  For example,
% % \begin{verbatim}
% %    \citet{hasselmo} investigated\dots
% % \end{verbatim}
% % produces
% % \begin{quote}
% %   Hasselmo, et al.\ (1995) investigated\dots
% % \end{quote}
% % If you wish to load the \verb+natbib+ package with options, you may add the
% % following before loading the \verb+neurips_2020+ package:
% % \begin{verbatim}
% %    \PassOptionsToPackage{options}{natbib}
% % \end{verbatim}
% % If \verb+natbib+ clashes with another package you load, you can add the optional
% % argument \verb+nonatbib+ when loading the style file:
% % \begin{verbatim}
% %    \usepackage[nonatbib]{neurips_2020}
% % \end{verbatim}
% % As submission is double blind, refer to your own published work in the third
% % person. That is, use ``In the previous work of Jones et al.\ [4],'' not ``In our
% % previous work [4].'' If you cite your other papers that are not widely available
% % (e.g., a journal paper under review), use anonymous author names in the
% % citation, e.g., an author of the form ``A.\ Anonymous.''
% % \subsection{Footnotes}
% % Footnotes should be used sparingly.  If you do require a footnote, indicate
% % footnotes with a number\footnote{Sample of the first footnote.} in the
% % text. Place the footnotes at the bottom of the page on which they appear.
% % Precede the footnote with a horizontal rule of 2~inches (12~picas).
% % Note that footnotes are properly typeset \emph{after} punctuation
% % marks.\footnote{As in this example.}
% % \subsection{Figures}
% % \begin{figure}
% %   \centering
% %   \fbox{\rule[-.5cm]{0cm}{4cm} \rule[-.5cm]{4cm}{0cm}}
% %   \caption{Sample figure caption.}
% % \end{figure}
% % All artwork must be neat, clean, and legible. Lines should be dark enough for
% % purposes of reproduction. The figure number and caption always appear after the
% % figure. Place one line space before the figure caption and one line space after
% % the figure. The figure caption should be lower case (except for first word and
% % proper nouns); figures are numbered consecutively.
% % You may use color figures.  However, it is best for the figure captions and the
% % paper body to be legible if the paper is printed in either black/white or in
% % color.
% % \subsection{Tables}
% % All tables must be centered, neat, clean and legible.  The table number and
% % title always appear before the table.  See Table~\ref{sample-table}.
% % Place one line space before the table title, one line space after the
% % table title, and one line space after the table. The table title must
% % be lower case (except for first word and proper nouns); tables are
% % numbered consecutively.
% % Note that publication-quality tables \emph{do not contain vertical rules.} We
% % strongly suggest the use of the \verb+booktabs+ package, which allows for
% % typesetting high-quality, professional tables:
% % \begin{center}
% %   \url{https://www.ctan.org/pkg/booktabs}
% % \end{center}
% % This package was used to typeset Table~\ref{sample-table}.
% % \begin{table}
% %   \caption{Sample table title}
% %   \label{sample-table}
% %   \centering
% %   \begin{tabular}{lll}
% %     \toprule
% %     \multicolumn{2}{c}{Part}                   \\
% %     \cmidrule(r){1-2}
% %     Name     & Description     & Size ($\mu$m) \\
% %     \midrule
% %     Dendrite & Input terminal  & $\sim$100     \\
% %     Axon     & Output terminal & $\sim$10      \\
% %     Soma     & Cell body       & up to $10^6$  \\
% %     \bottomrule
% %   \end{tabular}
% % \end{table}
% % \section{Final instructions}
% % Do not change any aspects of the formatting parameters in the style files.  In
% % particular, do not modify the width or length of the rectangle the text should
% % fit into, and do not change font sizes (except perhaps in the
% % \textbf{References} section; see below). Please note that pages should be
% % numbered.
% % \section{Preparing PDF files}
% % Please prepare submission files with paper size ``US Letter,'' and not, for
% % example, ``A4.''
% % Fonts were the main cause of problems in the past years. Your PDF file must only
% % contain Type 1 or Embedded TrueType fonts. Here are a few instructions to
% % achieve this.
% % \begin{itemize}
% % \item You should directly generate PDF files using \verb+pdflatex+.
% % \item You can check which fonts a PDF files uses.  In Acrobat Reader, select the
% %   menu Files$>$Document Properties$>$Fonts and select Show All Fonts. You can
% %   also use the program \verb+pdffonts+ which comes with \verb+xpdf+ and is
% %   available out-of-the-box on most Linux machines.
% % \item The IEEE has recommendations for generating PDF files whose fonts are also
% %   acceptable for NeurIPS. Please see
% %   \url{http://www.emfield.org/icuwb2010/downloads/IEEE-PDF-SpecV32.pdf}
% % \item \verb+xfig+ "patterned" shapes are implemented with bitmap fonts.  Use
% %   "solid" shapes instead.
% % \item The \verb+\bbold+ package almost always uses bitmap fonts.  You should use
% %   the equivalent AMS Fonts:
% % \begin{verbatim}
% %    \usepackage{amsfonts}
% % \end{verbatim}
% % followed by, e.g., \verb+\mathbb{R}+, \verb+\mathbb{N}+, or \verb+\mathbb{C}+
% % for $\mathbb{R}$, $\mathbb{N}$ or $\mathbb{C}$.  You can also use the following
% % workaround for reals, natural and complex:
% % \begin{verbatim}
% %    \newcommand{\RR}{I\!\!R} %real numbers
% %    \newcommand{\Nat}{I\!\!N} %natural numbers
% %    \newcommand{\CC}{I\!\!\!\!C} %complex numbers
% % \end{verbatim}
% % Note that \verb+amsfonts+ is automatically loaded by the \verb+amssymb+ package.
% % \end{itemize}
% % If your file contains type 3 fonts or non embedded TrueType fonts, we will ask
% % you to fix it.
% % \subsection{Margins in \LaTeX{}}
% % Most of the margin problems come from figures positioned by hand using
% % \verb+\special+ or other commands. We suggest using the command
% % \verb+\includegraphics+ from the \verb+graphicx+ package. Always specify the
% % figure width as a multiple of the line width as in the example below:
% % \begin{verbatim}
% %    \usepackage[pdftex]{graphicx} ...
% %    \includegraphics[width=0.8\linewidth]{myfile.pdf}
% % \end{verbatim}
% % See Section 4.4 in the graphics bundle documentation
% % (\url{http://mirrors.ctan.org/macros/latex/required/graphics/grfguide.pdf})
% % A number of width problems arise when \LaTeX{} cannot properly hyphenate a
% % line. Please give LaTeX hyphenation hints using the \verb+\-+ command when
% % necessary.
% % \section*{Broader Impact}
% % Authors are required to include a statement of the broader impact of their work, including its ethical aspects and future societal consequences.
% % Authors should discuss both positive and negative outcomes, if any. For instance, authors should discuss a)
% % who may benefit from this research, b) who may be put at disadvantage from this research, c) what are the consequences of failure of the system, and d) whether the task/method leverages
% % biases in the data. If authors believe this is not applicable to them, authors can simply state this.
% % Use unnumbered first level headings for this section, which should go at the end of the paper. {\bf Note that this section does not count towards the eight pages of content that are allowed.}
% % \begin{ack}
% % Use unnumbered first level headings for the acknowledgments. All acknowledgments
% % go at the end of the paper before the list of references. Moreover, you are required to declare
% % funding (financial activities supporting the submitted work) and competing interests (related financial activities outside the submitted work).
% % More information about this disclosure can be found at: \url{https://neurips.cc/Conferences/2020/PaperInformation/FundingDisclosure}.
% % Do {\bf not} include this section in the anonymized submission, only in the final paper. You can use the \texttt{ack} environment provided in the style file to autmoatically hide this section in the anonymized submission.
% % \end{ack}
% \pagebreak
\bibliographystyle{apalike}
\bibliography{biblio}
% \section*{References}
% % References follow the acknowledgments. Use unnumbered first-level heading for
% % the references. Any choice of citation style is acceptable as long as you are
% % consistent. It is permissible to reduce the font size to \verb+small+ (9 point)
% % when listing the references.
% % {\bf Note that the Reference section does not count towards the eight pages of content that are allowed.}
% % \medskip
% % \small
% % [1] Alexander, J.A.\ \& Mozer, M.C.\ (1995) Template-based algorithms for
% % connectionist rule extraction. In G.\ Tesauro, D.S.\ Touretzky and T.K.\ Leen
% % (eds.), {\it Advances in Neural Information Processing Systems 7},
% % pp.\ 609--616. Cambridge, MA: MIT Press.
% % [2] Bower, J.M.\ \& Beeman, D.\ (1995) {\it The Book of GENESIS: Exploring
% %   Realistic Neural Models with the GEneral NEural SImulation System.}  New York:
% % TELOS/Springer--Verlag.
% % [3] Hasselmo, M.E., Schnell, E.\ \& Barkai, E.\ (1995) Dynamics of learning and
% % recall at excitatory recurrent synapses and cholinergic modulation in rat
% % hippocampal region CA3. {\it Journal of Neuroscience} {\bf 15}(7):5249-5262.

% \section{Further Directions}

% \begin{itemize}
%     \item Use Dirichlet distribution for arms (Need Boojum distribution, a conjugate prior of dirichlet distribution)
%     \item Check this article for Boojum distribution \url{https://arxiv.org/abs/1811.05266}
% \end{itemize}

\end{document}
 
 
 

